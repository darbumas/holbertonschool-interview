\documentclass{article}
\usagepackage{algorithm}
\usagepackage[noend]{algpseudocode}

\begin{document}

\section*{Climbing Stairs}

To solve the Climbing Stairs problem, we can use dynamic programming. We can
break down the problem into smaller subproblems and use memoization to store the
results of these subproblems for reuse.

\subsection*{Approach}

\begin{algorithm}[H]
\caption{Climbing Stairs}
\begin{algorithmic}[1]
\Function{ClimbStairs}{$n$}
	\If{$n = 1$}
		\State \Return $1$
	\EndIf
	\If{$n = 2$}
		\State \Return $2$
	\EndIf

	\State $memo \gets$ list of size $n+1$ initialized to $0$
	\State $memo[1] \gets 1$
	\State $memo[2] \gets 2$

	\For{$i \gets 3$ \textbf{to} $n$}
		\State $memo[i] \gets memo[i-1] + memo[i-2]$
	\EndFor

	\State \Return $memo[n]$

\EndFunction
\end{algorithmic}
\end{algorithm}

The time complexity of this algorithm is $O(n)$ because we have to iterate from
$3$ to $n$ once to calculate the number of ways to reach each step.

The space complexity is also $O(n)$ because we have to store the results of each
subproblem in the memoization list.

\subsection*{Sample Code}

Here's an example of how to use the \texttt{ClimbStairs} function:

\begin{verbatim}
>>> n = 5
>>> result = ClimbStairs(n)
>>> print(result) # Output: 8
\end{verbatim}

\end{document}
